% LaTeX Curriculum Vitae Template
%
% Copyright (C) 2004-2009 Jason Blevins <jrblevin@sdf.lonestar.org>
% http://jblevins.org/projects/cv-template/
%
% You may use use this document as a template to create your own CV
% and you may redistribute the source code freely. No attribution is
% required in any resulting documents. I do ask that you please leave
% this notice and the above URL in the source code if you choose to
% redistribute this file.

\documentclass[letterpaper]{article}

\usepackage{hyperref}
\usepackage{geometry}
\usepackage{sectsty}

% Set your name here
\def\name{Rebecca L. Martinez}

% Replace this with a link to your CV if you like, or set it empty
% (as in \def\footerlink{}) to remove the link in the footer:
\def\footerlink{https://rebeccalmartinez.github.io/personal_website/experience.html}

% The following metadata will show up in the PDF properties
\hypersetup{
  colorlinks = true,
  urlcolor = black,
  pdfauthor = {\name},
  pdfkeywords = {environmental science, field research, education, UCSB, mentorship},
  pdftitle = {\name: Curriculum Vitae},
  pdfsubject = {Curriculum Vitae},
  pdfpagemode = UseNone
}

\geometry{
  body={6.5in, 8.5in},
  left=1.0in,
  top=1.25in
}

% Customize page headers
\pagestyle{myheadings}
\markright{\name}
\thispagestyle{empty}

% Custom section fonts (using sectsty pkg)
\sectionfont{\rmfamily\mdseries\Large}
\subsectionfont{\rmfamily\mdseries\itshape\large}

% Other possible font commands include:
% \ttfamily for teletype,
% \sffamily for sans serif,
% \bfseries for bold,
% \scshape for small caps,
% \normalsize, \large, \Large, \LARGE sizes.

% Don't indent paragraphs.
\setlength\parindent{0em}

% Make lists without bullets
\renewenvironment{itemize}{
  \begin{list}{}{
    \setlength{\leftmargin}{1.5em}
  }
}{
  \end{list}
}

\newenvironment{biblist}{%
   \begin{list}{}{%
     \setlength{\labelwidth}{0pt}%
     \setlength{\labelsep}{1em}%
     \setlength{\leftmargin}{2em}%
     \setlength{\itemindent}{-1em}%
   }
}{\end{list}}


\begin{document}

% Place name centered and bold
\centerline{\huge \bf \name}
\begin{center}
  \emph{Curriculum Vitae}
\end{center}


\vspace{0.25in}

\begin{minipage}{0.55\linewidth}
Undergraduate Researcher and Student Assistant\\
University of California, Santa Barbara\\
Office of Education Partnerships\\
Santa Barbara, CA
\end{minipage}
\begin{minipage}{0.45\linewidth}
  \begin{tabular}{ll}
    e-mail: & \href{mailto:rebecca_martinez@ucsb.edu}{\tt rebecca\_martinez@ucsb.edu} \\
            & \href{mailto:martinez.rl@hotmail.com}{\tt martinez.rl@hotmail.com} \\
    website: & \href{https://rebeccalmartinez.github.io/personal_website/}{\url{https://rebeccalmartinez.github.io/personal_website/}}

  \end{tabular}
\end{minipage}


%-----------------------------------
% Education
%-----------------------------------
\section*{Education}


\begin{itemize}
  \item B.S. Environmental Studies, University of California, Santa Barbara  
  \emph{Expected 2026}

  \item A.S. Biology, Allan Hancock College  
  \emph{with honors, May 2024}

  \item A.A. Biology, Allan Hancock College  
  \emph{with honors, Dec 2023}

  \item A.A. Mathematics and Science, Allan Hancock College  
  \emph{with high honors, May 2021}

  \item IT Fundamentals Certificate, Allan Hancock College  
  \emph{May 2024}
\end{itemize}


%-----------------------------------
% Research Experience
%-----------------------------------
\section*{Research Experience}

\begin{biblist}
\item Peer Mentor, ERES Field Workshop, Santa Cruz Island Reserve (UCNRS) \\
UCSB–Smithsonian Scholars Program (Summer 2025)  
\begin{itemize}
  \item Trained students in the use of field sensors (camera traps, Audiomoths) and GPS
  \item Supported data collection using FAIR data practices, orienteering, and bioacoustic survey protocols
\end{itemize}

\item Undergraduate Researcher, 199RA \\
UC Santa Barbara – under Dr. Iris Holzer (Winter 2025–Present)  
\emph{The Role of Mammalian Bioturbators in Soil Formation and Elemental Cycling: Comparing rock-derived N release across the Channel Islands and the Transverse Ranges using an undergraduate-led research framework}  
\begin{itemize}
  \item Collected and processed soil and rock samples
  \item Prepped samples via rock saw for N-release assays and isotope analysis
  \item Drafted literature review and developed learning resources
  \item Planning AGU Fall Meeting presentation
\end{itemize}

\item Research Assistant, UCNRS Sedgwick Reserve (Spring 2025–Present) \\
\emph{Soil Carbon Stabilization Study}  
\begin{itemize}
  \item Assisted with pit digging, bulk density sampling, and soil auger collection
  \item Performed USDA soil profile descriptions: color, roots, pores, aggregates
  \item Applied safe sampling protocols and helped maintain positive morale on long field days
\end{itemize}

\item Student Assistant, UCSB Office of Education Partnerships (Fall 2024–Present)  
\begin{itemize}
  \item Revised protocols for fog water collection devive, 'Fog Harp', project
  \item Collected and managed camera trap data for the Santa Cruz Island Skunk project
  \item Maintained inventory and provided peer mentorship and support
\end{itemize}

\item Discussion Leader (Intern), UCSB–Smithsonian Data Bootcamp \\
Smithsonian Data Science Lab – Dr. Alexander White (Summer 2023)  
\begin{itemize}
  \item Facilitated data science instruction in R via RStudio/PositCloud as part of a Smithsonian Data Science Lab internship
  \item Introduced biodiversity-focused datasets to undergraduates, emphasizing reproducible workflows and visual data communication
\end{itemize}

\item Participant, Advanced Field Research and Data Science Training, Smithsonian Tropical Research Institute (STRI) Panama \\
UCSB–Smithsonian Scholars Program (Summer 2023)  
\begin{itemize}
  \item Completed two-week field methods course and advanced data science training in RStudio/PositCloud
  \item Shadowed Smithsonian researchers to learn ecological field methods and ongoing research at STRI and participated in guided ecotourism
  \item Developed an a priori question and completed a group research project: \textit{Comparison of Species Richness and Functional Richness on Barro Colorado Island and Pipeline Road in Panama}
\end{itemize}

\item Participant, ERES Field Research Workshop, Santa Cruz Island \\
UCSB–Smithsonian Scholars Program (Summer 2023)
\end{biblist}



%-----------------------------------
% Teaching and Mentorship Experience
%-----------------------------------
\section*{Teaching and Mentorship}

\begin{biblist}

\item Discussion Leader, UCSB–Smithsonian Data Bootcamp (Summer 2024, 2025)\\
Guided undergraduates in using RStudio/PositCloud for data manipulation and visualization, emphasizing collaborative learning, equitable access, and open science principles.

\item Teaching Assistant, Human Anatomy Open Lab, Allan Hancock College (2023–2024)\\
Instructor: Alicia Fox\\
Managed Friday lab sessions including setup and breakdown; developed visual study guides, led office hours, and supported human cadaver lab activities.

\item Embedded Tutor (TA Support Role), Allan Hancock College (2020–2024)\\
Supported multiple courses by creating review materials, leading exam preparation sessions, holding office hours, and assisting with grading.\\
Courses and instructors include: 
\begin{itemize}
  \item Botany — Wendy Hadley
  \item Cellular Biology and Physiology — Ashley Wise
  \item Introductory and General Chemistry — Dr. James Houlis
  \item Human Anatomy — Dr. Len Miyahara
  \item Microbiology (online) — Dr. Tim Doyle
\end{itemize}

\item Lead Tutor and Tutor, Allan Hancock College Academic Resource Center \& STEM Center (2019–2024)\\
Delivered one-on-one and group tutoring in biology, chemistry, math, and general education. Promoted to Lead Tutor in 2020, where I trained new tutors and hosted workshops on study skills and time management. Assisted with staff and faculty onboarding orientations.

\end{biblist}


%-----------------------------------
% Leadership and Service
%-----------------------------------
\section*{Leadership and Service}

\begin{biblist}

\item Outreach, UCSB Student Veterans Association (SVA) and She Raised Her Hand (2024--Present)\\
Organized networking events to connect veteran students and women veterans from both UCSB’s SVA and She Raised Her Hand networks.

\item Science Outreach Volunteer, Allan Hancock College Life \& Physical Sciences Department (2022--2024)\\
Designed and facilitated hands-on science activities for local K--12 students; created educational materials and coordinated event logistics.

\item Student Representative, Allan Hancock College Hiring Committees (2023--2024)\\
Participated as student representative in full-time faculty and staff selection committees.

\item Student Representative, Allan Hancock College Title V Committee (2021--2023)\\
Advocated for student needs in planning academic support programs; collaborated with faculty and committee members to improve and implement student support initiatives.

\item Vice President and Community Service Chair, AGS Aquarius Honor Society (2019--2021)\\
Organized community service projects and fundraisers; developed and managed an online system for tracking service points.

\item United States Marine Corps (2005--2010)\\
Completed specialized training in land navigation, survival skills, swim qualification, and CBRN defense; developed leadership and operational skills applicable to academic research and fieldwork.

\end{biblist}


%-----------------------------------
% Presentations
%-----------------------------------
\section*{Presentations}

\begin{biblist}

\item \textit{Learning from the Ground Up: Mammalian Bioturbators, Soil, \& Undergraduate Education}, UCSB Environmental Studies 55th Anniversary Symposium, Santa Barbara, CA, April 2025.

\item Student Speaker, Allan Hancock Foundation Honors Gala, Santa Maria, CA, May 2024.

\item \textit{Comparison of Species Richness \& Functional Richness Between Barro Colorado Island and Pipeline Road in Panama}, Cal Poly Summer Internship Research Symposium, San Luis Obispo, CA, August 2023.

\item \textit{Comparison of Species Richness \& Functional Richness Between Barro Colorado Island and Pipeline Road in Panama}, UCSB Fall Undergraduate Research Showcase, Santa Barbara, CA, November 2023.

\end{biblist}


%-----------------------------------
% Awards & Scholarships
%-----------------------------------
\section*{Awards and Scholarships}

\begin{biblist}
\item UCNRS Field Science Fellowship, 2025
\item Wenger Scholarship, 2025
\item Gene \& Susan Lucas Undergraduate Research Fund, 2025
\item G.\ Allan Hancock Scholarship, 2024
\item Allan Hancock MESA Scholarship, 2023
\item Brander Single Parent Scholarship, 2022, 2023, 2024
\item Joyce Dendo Veteran Scholarship (Inaugural), 2021
\item Kathleen D.\ Loly Service Scholarship, 2021
\item Joan Semelsberger Scholarship, 2021, 2022, 2023
\item Virginia B.\ Martinez Memorial Scholarship, 2021
\item AGS Aquarius Honor Society Leadership Award, 2021
\end{biblist}

%-----------------------------------
% Programs & Affiliations
%-----------------------------------
\section*{Programs and Affiliations}

\begin{itemize}
\item Bren–Environmental Studies Fellows Program, 2025–2026 cohort
\item Tau Sigma National Honor Society, Member since 2025
\item American Geophysical Union (AGU), Member since 2025
\item UCSB Promise Scholars Program, Member since 2024
\item UCSB Letters \& Science Honors Program, Member since 2024
\item UCSB Student Veterans Association (SVA), Member since 2024
\item UCSB–Smithsonian Scholars Program (ERES participant 2023; discussion leader/mentor 2024–2025), Member since 2023
\item Alpha Gamma Sigma Honor Society, (Vice President 2019-2021 and Community Service Chair 2022-2023), Member since 2019
\item Association of Colleges for Tutoring \& Learning Assistance (ACTLA), Member since 2021
\item Lompoc Veteran Advisory Committee (LVAC), Member since 2017
\end{itemize}

 
\bigskip

% Footer
\begin{center}
  \begin{footnotesize}
    Last updated: \today \\
    \href{\footerlink}{\texttt{\footerlink}}
  \end{footnotesize}
\end{center}

\end{document}