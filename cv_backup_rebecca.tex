% LaTeX Curriculum Vitae Template
%
% Copyright (C) 2004-2009 Jason Blevins <jrblevin@sdf.lonestar.org>
% http://jblevins.org/projects/cv-template/
%
% You may use use this document as a template to create your own CV
% and you may redistribute the source code freely. No attribution is
% required in any resulting documents. I do ask that you please leave
% this notice and the above URL in the source code if you choose to
% redistribute this file.

\documentclass[letterpaper]{article}

\usepackage{hyperref}
\usepackage{geometry}
\usepackage{sectsty}

% Set your name here
\def\name{Rebecca L. Martinez}

% Replace this with a link to your CV if you like, or set it empty
% (as in \def\footerlink{}) to remove the link in the footer:
\def\footerlink{https://rebeccalmartinez.github.io/personal_website/experience.html}

% The following metadata will show up in the PDF properties
\hypersetup{
  colorlinks = true,
  urlcolor = black,
  pdfauthor = {\name},
  pdfkeywords = {environmental science, field research, education, UCSB, mentorship},
  pdftitle = {\name: Curriculum Vitae},
  pdfsubject = {Curriculum Vitae},
  pdfpagemode = UseNone
}

\geometry{
  margin=0.75in
}


\newcommand{\sectionheader}[1]{%
  \noindent\textbf{\Large #1}\\[-1.5ex]
  \noindent\rule{\linewidth}{0.4pt}\\[-1.5ex]
}


% Customize page headers
\pagestyle{myheadings}
\markright{\name}
\thispagestyle{empty}

% Custom section fonts (using sectsty pkg)
\sectionfont{\rmfamily\mdseries\Large}
\subsectionfont{\rmfamily\mdseries\itshape\large}

% Other possible font commands include:
% \ttfamily for teletype,
% \sffamily for sans serif,
% \bfseries for bold,
% \scshape for small caps,
% \normalsize, \large, \Large, \LARGE sizes.

% Don't indent paragraphs.
\setlength\parindent{0em}

% Make lists without bullets
\renewenvironment{itemize}{
  \begin{list}{}{
    \setlength{\leftmargin}{1.5em}
  }
}{
  \end{list}
}

\newenvironment{biblist}{%
   \begin{list}{}{%
     \setlength{\labelwidth}{0pt}%
     \setlength{\labelsep}{1em}%
     \setlength{\leftmargin}{2em}%
     \setlength{\itemindent}{-1em}%
   }
}{\end{list}}


\begin{document}

% Place name centered and bold
\begin{center}
  {\huge \textbf{Rebecca L. Martinez}}\\[0.2cm]
  Undergraduate Researcher and Student Assistant\\
  Environmental Studies Program • University of California, Santa Barbara\\
  Santa Barbara, CA\\[0.2cm]
  \href{mailto:rebecca_martinez@ucsb.edu}{rebecca\_martinez@ucsb.edu} \quad • \quad
  \href{mailto:martinez.rl@hotmail.com}{martinez.rl@hotmail.com} \quad • \quad
  \href{https://rebeccalmartinez.github.io/personal_website/}{https://rebeccalmartinez.github.io/personal_website/}
\end{center}


\vspace{0.1in}

%-----------------------------------
% Education
%-----------------------------------

\sectionheader{Education}


\begin{itemize}
  \item B.S. Environmental Studies, University of California, Santa Barbara  
  \emph{Expected June 2026}

  \item A.S. Biology, Allan Hancock College  
  \emph{with honors, May 2024}

  \item A.A. Biology, Allan Hancock College  
  \emph{with honors, December 2023}

  \item A.A. Mathematics and Science, Allan Hancock College  
  \emph{with high honors, May 2021}

  \item IT Fundamentals Certificate, Allan Hancock College  
  \emph{May 2024}
\end{itemize}


%-----------------------------------
% Research Experience
%-----------------------------------
\item \textbf{Undergraduate Researcher, UC Santa Barbara – under Dr. Iris Holzer} \hfill January 2025–Present \\
\emph{Geologic Nitrogen Inputs to Semi-Arid Soils with and without Mammalian Bioturbation}

\begin{itemize}
  \item Collected and prepared soil and rock samples for nitrogen and isotope analysis
  \item Trained in rock saw and shatterbox use; prepared powdered samples in foil capsules for lab testing
  \item Reviewed scientific literature and helped develop student learning materials
\end{itemize}


\item \textbf{Research Assistant} \\ 
UC Davis – UCNRS Sedgwick Reserve Soil Carbon Project (March 2025–Present) \\
\emph{Soil Carbon Stabilization Study}
\begin{itemize}
  \item Assisted with pit digging, bulk density sampling, and soil auger collection
  \item Performed USDA soil profile descriptions: color, roots, pores, aggregates
  \item Applied safe sampling protocols and helped maintain positive morale on long field days
\end{itemize}

\item \textbf{Student Assistant} \\
UCSB Office of Education Partnerships (September 2024–Present)  
\begin{itemize}
  \item Revised protocols for fog water collection devive, 'Fog Harp', project
  \item Collected and managed camera trap data for the Santa Cruz Island Skunk project
  \item Maintained inventory and provided peer mentorship and support
\end{itemize}

\item \textbf{Peer Mentor}, UCSB–Smithsonian Scholars Program (June–September 2025)   
\begin{itemize}
  \item Mentored students during the ERES Field Workshop at Santa Cruz Island Reserve (UCNRS)
  \item Trained students on field sensors including camera traps, Audiomoths, and GPS
  \item Supported ecological data collection using FAIR data practices
\end{itemize}

\item \textbf{Discussion Leader}, Smithsonian Data Science Lab \\
UCSB–Smithsonian Scholars Program (June–September 2024, 2025) 
\begin{itemize}
  \item Led data science instruction in R using RStudio and PositCloud
  \item Emphasized reproducible workflows and data visualization techniques
  \item Supported collaborative learning and student engagement
\end{itemize}

\item \textbf{Participant}, UCSB–Smithsonian Scholars Program (June–September 2023)  
\begin{itemize}
  \item Completed advanced field methods and data science training at Smithsonian Tropical Research Institute (STRI), Panama, including RStudio/PositCloud instruction
  \item Shadowed researchers to learn ecological field techniques and ongoing projects at STRI; participated in guided ecotourism
  \item Developed an a priori research question and collaborated on a group project: \textit{Comparison of Species Richness and Functional Richness on Barro Colorado Island and Pipeline Road in Panama}
  \item Attended ERES Field Research Workshop at Santa Cruz Island Reserve (UCNRS), focusing on field sensor use and ecological data collection
\end{itemize}


%-----------------------------------
% Teaching and Mentorship Experience
\sectionheader{Teaching and Mentorship}

\begin{biblist}

\item \textbf{Discussion Leader, UCSB–Smithsonian Data Bootcamp (Summer 2024, 2025)}\\
Guided undergraduates in using RStudio/PositCloud for data manipulation and visualization, emphasizing collaborative learning, equitable access, and open science principles.

\item \textbf{Teaching Assistant}, Human Anatomy Open Lab, Allan Hancock College (May 2023 – May 2024)\\
Instructor: Alicia Fox\\
Managed Friday lab sessions, including setup and breakdown; developed visual study guides, led office hours, and supported human cadaver lab activities.

\item \textbf{Embedded Tutor}, Allan Hancock College (2020–2024)\\
Collaborated with instructors to support student success through review sessions, office hours, grading assistance, and tailored study materials.\\
Courses supported: Botany, Chemistry (Introductory and General), Human Anatomy and Physiology, Cellular Biology, and Microbiology (online).

\item \textbf{Lead Tutor and Tutor}, Allan Hancock College Academic Resource Center \& STEM Center (2019–2024)\\
Delivered one-on-one and group tutoring in biology, chemistry, math, and general education. Promoted to Lead Tutor in 2020; trained new tutors and hosted workshops on study skills and time management. Supported staff and faculty onboarding orientations.

\end{biblist}


%-----------------------------------
% Leadership and Service
%-----------------------------------
\sectionheader{Leadership and Service}

\begin{biblist}

\item \textbf{Outreach Volunteer}, UCSB Veteran Support Center and She Raised Her Hand (2024--Present)\\
Assisted with organizing networking events to connect veteran students and women veterans within UCSB’s communities.

\item \textbf{Science Outreach Volunteer}, Allan Hancock College Life \& Physical Sciences Department (2022--2024)\\
Designed and facilitated hands-on science activities for local K–12 students; created educational materials and coordinated event logistics.

\item \textbf{Student Representative}, Allan Hancock College Hiring Committees (2023--2024)\\
Represented student perspectives during faculty and staff hiring processes.

\item \textbf{Student Representative}, Allan Hancock College Title V Committee (2021--2023)\\
Advocated for student needs in academic program planning; collaborated with faculty and committee members to improve student support services.

\item \textbf{Vice President and Community Service Chair}, AGS Aquarius Honor Society (2019--2021)\\
Organized community service projects and fundraisers; developed and maintained an online system for tracking service points.

\item \textbf{United States Marine Corps}, (2005--2010)\\
Completed specialized training in land navigation, survival skills, swim qualification, and CBRN defense; developed leadership and operational skills applicable to research and fieldwork.


\end{biblist}


%-----------------------------------
% Presentations
%-----------------------------------
\sectionheader{Presentations}

\begin{biblist}

\item Abstract submitted for presentation at AGU Fall Meeting, December 2025: \textit{Geologic Nitrogen Inputs to Semi-Arid Soils with and without Mammalian Bioturbation}.

\item \textit{Learning from the Ground Up: Mammalian Bioturbators, Soil, \& Undergraduate Education}, UCSB Environmental Studies 55th Anniversary Symposium, Santa Barbara, CA, April 2025.

\item Student Speaker, Allan Hancock Foundation Honors Gala, Santa Maria, CA, May 2024.

\item \textit{Comparison of Species Richness \& Functional Richness Between Barro Colorado Island and Pipeline Road in Panama}, Cal Poly Summer Internship Research Symposium, San Luis Obispo, CA, August 2023.

\item \textit{Comparison of Species Richness \& Functional Richness Between Barro Colorado Island and Pipeline Road in Panama}, UCSB Fall Undergraduate Research Showcase, Santa Barbara, CA, November 2023.

\end{biblist}


%-----------------------------------
% Awards & Scholarships
%-----------------------------------
\sectionheader{Awards and Scholarships}

\begin{biblist}
\item UCNRS Field Science Fellowship, 2025
\item Wenger Scholarship, 2025
\item Gene \& Susan Lucas Undergraduate Research Fund, 2025
\item G.\ Allan Hancock Scholarship, 2024
\item Allan Hancock MESA Scholarship, 2023
\item Brander Single Parent Scholarship, 2022, 2023, 2024
\item Joyce Dendo Veteran Scholarship (Inaugural), 2021
\item Kathleen D.\ Loly Service Scholarship, 2021
\item Joan Semelsberger Scholarship, 2021, 2022, 2023
\item Virginia B.\ Martinez Memorial Scholarship, 2021
\item AGS Aquarius Honor Society Leadership Award, 2021
\end{biblist}

%-----------------------------------
% Programs & Affiliations
%-----------------------------------
\sectionheader{Programs and Affiliations}

\begin{itemize}
\item Bren–Environmental Studies Fellows Program, 2025–2026 cohort
\item Tau Sigma National Honor Society, Member since 2025
\item American Geophysical Union (AGU), Member since 2025
\item UCSB Promise Scholars Program, Member since 2024
\item UCSB Letters \& Science Honors Program, Member since 2024
\item UCSB Student Veterans Association (SVA), Member since 2024
\item UCSB–Smithsonian Scholars Program (ERES participant 2023; discussion leader/mentor 2024–2025), Member since 2023
\item Alpha Gamma Sigma Honor Society, (Vice President 2019-2021 and Community Service Chair 2022-2023), Member since 2019
\item Association of Colleges for Tutoring \& Learning Assistance (ACTLA), Member since 2021
\item Lompoc Veteran Advisory Committee (LVAC), Member since 2017
\end{itemize}

 
\bigskip

% Footer
\begin{center}
  \begin{footnotesize}
    Last updated: \today \\
    \href{\footerlink}{\texttt{\footerlink}}
  \end{footnotesize}
\end{center}

\end{document}